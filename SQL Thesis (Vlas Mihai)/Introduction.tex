\section*{Introduction}
\phantomsection

\textbf{SQL} stands for Structured Query Language, is a domain-specific language used in programming and designed for managing data held in a relational database management system (RDBMS), or for stream processing in a relational data stream management system. Originally based upon relational algebra and tuple relational calculus, SQL consists of a data definition language, data manipulation language, and data control language. The scope of SQL includes data insert, query, update and delete, schema creation and modification, and data access control. Although SQL is often described as, and to a great extent is, a declarative language (4GL), it also includes procedural elements.

SQL was one of the first commercial languages for Edgar F. Codd's relational model, as described in his influential 1970 paper, "A Relational Model of Data for Large Shared Data Banks." Despite not entirely adhering to the relational model as described by Codd, it became the most widely used database language.

SQL became a standard of the American National Standards Institute (ANSI) in 1986, and of the International Organization for Standardization (ISO) in 1987. Since then, the standard has been revised to include a larger set of features. Despite the existence of such standards, most SQL code is not completely portable among different database systems without adjustments.
The SQL language is subdivided into several language elements, including:

Clauses, which are constituent components of statements and queries. (In some cases, these are optional.)
Expressions, which can produce either scalar values, or tables consisting of columns and rows of data
Predicates, which specify conditions that can be evaluated to SQL three-valued logic (3VL) (true/false/unknown) or Boolean truth values and are used to limit the effects of statements and queries, or to change program flow.
Queries, which retrieve the data based on specific criteria. This is an important element of SQL.
Statements, which may have a persistent effect on schemata and data, or may control transactions, program flow, connections, sessions, or diagnostics.
SQL statements also include the semicolon (";") statement terminator. Though not required on every platform, it is defined as a standard part of the SQL grammar.
Insignificant whitespace is generally ignored in SQL statements and queries, making it easier to format SQL code for readability.
\clearpage