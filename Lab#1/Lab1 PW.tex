%----------------------------------------------------------------------------------------
% PACKAGES AND DOCUMENT CONFIGURATIONS
%----------------------------------------------------------------------------------------

  \documentclass[12pt]{article}

  \usepackage{hyperref}
  \usepackage{fancyhdr} % Required for custom headers
  \usepackage{lastpage} % Required to determine the last page for the footer
 
  \usepackage[usenames,dvipsnames]{color} % Required for custom colors
  \usepackage{graphicx} % Required to insert images
  \usepackage{listings} % Required for insertion of code
  \usepackage{courier} % Required for the courier font
  \usepackage{lipsum} % Used for inserting dummy 'Lorem ipsum' text into the template
  \usepackage{wrapfig}
  \usepackage{color}
  \usepackage{lscape}
  \usepackage{pbox}

 \graphicspath{ {../Figures/} }% figures are taken from this folder
 \setlength{\intextsep}{3ex} % set space above and below in-line float (figures, tables etc)
\setlength{\floatsep}{2ex} % space between floats
%\setlength{\abovecaptionskip}{1ex} % space above float caption
%\setlength{\belowcaptionskip}{5ex} % space below float caption

% numerotation for figures and tables
\renewcommand{\thefigure}{\arabic{section}.\arabic{figure}}
\renewcommand{\thetable}{\arabic{section}.\arabic{table}}


  \setlength\parindent{0pt} % Removes all indentation from paragraphs
  \renewcommand{\labelenumi}{\alph{enumi}.} % Make numbering in the itemize environment by letter rather than number (e.g. section 6)

  % Margins
  \topmargin=-0.4in
  \evensidemargin=0.2in
  \oddsidemargin=-0.2in
  \textwidth=7.0in
  \textheight=9.0in
  % \headsep=0.25in

  % \linespread{1.1} % Line spacing

  \definecolor{dkgreen}{rgb}{0,0.6,0}
  \definecolor{gray}{rgb}{0.5,0.5,0.5}
  \definecolor{mauve}{rgb}{0.58,0,0.82}
  \definecolor{greyish}{rgb}{0.96,0.96,0.96}

  \lstset{
    backgroundcolor=\color{greyish},   % choose the background color; you must add \usepackage{color} or \usepackage{xcolor}
    frame=tblr,
    numbers=left,                       % where to put the line-numbers; possible values are (none, left, right)
    numbersep=5pt,                   % how far the line-numbers are from the code
    numberstyle=\tiny\color{mygray}, % the style that is used for the line-numbers
    language=Ruby,
    aboveskip=3mm,
    belowskip=3mm,
    showstringspaces=false,
    columns=flexible,
    basicstyle={\footnotesize\ttfamily},
    numbers=none,
    numberstyle=\tiny\color{gray},
    keywordstyle=\color{blue},
    commentstyle=\color{dkgreen},
    stringstyle=\color{mauve},
    breaklines=true,
    breakatwhitespace=true
    tabsize=1
  }

  \begin{document}
  \begin{titlepage}

%----------------------------------------------------------------------------------------
% TITLE PAGE INFORMATION
%----------------------------------------------------------------------------------------
 \newcommand{\HRule}{\rule{\linewidth}{0.5mm}} % Defines a new command for the horizontal lines, change thickness here
  \begin{center} % Center everything on the page

  %----------------------------------------------------------------------------------------
  % HEADING SECTIONS
  %----------------------------------------------------------------------------------------
  \textsc{\large Faculty of Computers, Informatics and Microelectronics}\\[0.5cm]
  \textsc{\large Technical University of Moldova}\\[1.2cm] % Name of your university/college
  \vspace{35 mm}
  \textsc{\Large Windows Programming}\\[0.5cm] % Major heading such as course name
  %\textsc{\large Laboratory work \#1-3}\\[0.5cm] % Minor heading such as course title
  \textsc{\large Laboratory work \# 1}\\[0.5cm] % Minor heading such as course title

  %----------------------------------------------------------------------------------------
  % TITLE SECTION
  %----------------------------------------------------------------------------------------
  \vspace{10 mm}
  \HRule \\[0.4cm]
  { \LARGE \bfseries Window. Window handling. Basic window’s form elements }\\[0.4cm] % Title of your document
  \HRule \\[1.5cm]

  %----------------------------------------------------------------------------------------
  % AUTHOR SECTION
  %----------------------------------------------------------------------------------------
      \vspace{30mm}

      \begin{minipage}{0.4\textwidth}
      \begin{flushleft} \large
      \emph{Authors:}\\
      Vlas \textsc{Mihai}
      \end{flushleft}
      \end{minipage}
      ~
      \begin{minipage}{0.4\textwidth}
      \begin{flushright} \large
      \emph{Supervisor:} \\
      Irina \textsc{Cojanu} % Supervisor's Name
      \end{flushright}
      \end{minipage}\\[4cm]

      \vspace{5 mm}
      % If you don't want a supervisor, uncomment the two lines below and remove the section above
      %\Large \emph{Author:}\\
      %John \textsc{Smith}\\[3cm] % Your name

      %----------------------------------------------------------------------------------------
      % DATE SECTION
      %----------------------------------------------------------------------------------------

      {\large \today}\\[3cm] % Date, change the \today to a set date if you want to be precise

      %----------------------------------------------------------------------------------------
      % LOGO SECTION
      %----------------------------------------------------------------------------------------

      %\includegraphics{Logo}\\[1cm] % Include a department/university logo - this will require the graphicx package

      %----------------------------------------------------------------------------------------

      \vfill % Fill the rest of the page with whitespace
      \end{center}
      \end{titlepage}

      % \newpage
      % \tableofcontents
      % \newpage

%----------------------------------------------------------------------------------------
% Introduction
%----------------------------------------------------------------------------------------

  \section{Introduction}

  \subsection{Topic}

  Window. Window handling. Basic window’s form elements

  \subsection{Task}

  \begin{itemize}
    \renewcommand{\labelitemi}{$\circ$}
    \item Create a Windows application
    \item In the middle of the window should be present the following text: "Done with Pride and Prejudice by student name". Replace student name with your name.
    \item On windows resize, text should reflow and be in window's middle (vertically and horizontally)
    \item Add 2 buttons to window: one with default styles, one with custom styles (size, background, text color, font family/size)
    \item Add 2 text elements to window: one with default styles, one with custom styles (size, background, text color, font family/size)
    \item Make elements to interact or change other elements (2 different interactions) (ex. on button click, change text element color or position)
    \item Change behavior of different window actions (at least 3). For ex.: on clicking close button, move window to a random location on display working space
  \end{itemize}

%----------------------------------------------------------------------------------------
% Implementation
%----------------------------------------------------------------------------------------

  \section{Results}




\subsection{Background Changing}
\begin{minipage}[b]{1.0\linewidth}
      \begin{center}
        \includegraphics[width=0.9\textwidth]{Lab11}
         \\ Fig. 1 Yellow background
      \end{center}
    \end{minipage}

    \begin{minipage}[b]{1.0\linewidth}
      \begin{center}
        \includegraphics[width=0.9\textwidth]{Lab12}
         \\ Fig. 2 Blue background
      \end{center}
    \end{minipage}
    \begin{minipage}[b]{1.0\linewidth}
      \begin{center}
        \includegraphics[width=0.9\textwidth]{Lab13}
         \\ Fig. 3 PopUp Window
      \end{center}
    \end{minipage}
    
        \begin{minipage}[b]{1.0\linewidth}
      \begin{center}
        \includegraphics[width=0.9\textwidth]{Lab14}
         \\ Fig.4 Font Changing
      \end{center}
    \end{minipage}
\section{Conclusion}

\textbf{For this laboratory work we made our basic window program with few buttons that apply to different fucntions for manipulating with the input from the keyboard. As you can see in the screenshots, there are several buttons that allows us to make different changes in our window, for font : "Times, Courier,Calibri", for Display and erase of some text and others. This code is made by Mihai Vlas but took a bit part from his colegues because of lack of time in this semester. He feels very sorry and will improve his attitude just like his grades in the 3rd year. No more volunteering, just writing a lot of code }

  
\end{document}